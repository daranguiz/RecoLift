\chapter{Introduction}

If we were to peruse the headlines on a technology news website such as \textit{The Verge} or \textit{ArsTechnica}, a vast majority of the posted articles would revolve around the new Apple Watch. Google has been pushing their new Android Wear platform for over a year now, and Pebble has been working for even longer, with their original Kickstarter launching on April 11, 2012 (REFERENCE). At least in the eyes of these device manufacturers, smartwatches comprise the next wave in mobile computing. 

Occurring concurrently is the recent public interest in personal analytics. Fitbit and Jawbone have become the two frontrunners in this space. Their product lines of fitness trackers primarily record data related to general purpose fitness, such as resting heart rate, number of steps taken per day, and calories burned. Both Google and Apple have also taken to this space, incorporating fitness tracking in their own wearable devices by including an optical heartrate on the undersides of their watches and various MEMS sensors onboard the watches themselves. They have also opened up new APIs to allow developers free access to their personal data stream (REFERENCE), enabling such applications as runner route tracking and sleep tracking (REFERENCE). One space has remained relatively empty of fitness tracking applications however, and that is strength training.

Strength training comprises of three distinct disciplines: weightlifting, powerlifting, and bodybuilding. Weightlifting involves two lifts only, the clean and jerk and the snatch. These two lifts are the only lifts among strength training exercises that are tested at the Olympics (REFERENCE). Both of these lifts are highly technical and not often performed by beginners, with the exception of CrossFit, a new lifting paradigm which starts beginners on high-repetition Olympic lifts. Powerlifting focuses on three lifts only, bench press, squat, and deadlift. Like weightlifting, the primary goal of powerlifting is to maximize the weight lifting among the three lifts. The main distinction, aside from the difference in lifts, is powerlifters tend to focus on raw strength, whereas weightlifters focus on speed. Finally, bodybuilding is significantly different than both weightlifting and powerlifting. Strength does not matter in bodybuilding, and as such, bodybuilders focus on a plethora of smaller, isolated lifts to improve their physique. It is not uncommon for a bodybuilder to spend two hours in the gym performing 30 or 40 sets at eight repetitions per set (REFERENCE). This can be problematic, as gym goers often neglect to record their lifts, leading to confusion during the next session in the gym. 

To this end, we have created an application based on the work of Morris et al. in \textit{RecoFit} (REFERENCE) which tracks exercises without user intervention using the commonly available Android Wear platform. 