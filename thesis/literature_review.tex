\chapter{Literature Review}

As mentioned, this work is based on \textit{RecoFit} by Morris et al. \cite{recofit}, which uses the intuition that exercise is distinctly periodic and can be well-discerned from non-exercise. They achieve 86\% segmentation accuracy and 98\% recognition accuracy using a custom-built arm-worn device which samples at 50Hz. Additionally, their computation is done on local desktop machine, eliminating the need for energy optimizations beyond the sampling rate of the IMU. Our solution uses readily-available hardware which has been gaining traction in the consumer space \cite{lit:applewatch, lit:androidwear} to expand the possible userbase. We also consider battery life optimizations on the Android devices while still maintaining comparable levels of accuracy, allowing a user to spend an hour at the gym and continue their day without smartwatch or smartphone recharging. Finally, because we utilize the Android smartwatch, we can make stronger assumptions about placement of the smartwatch. This enables us to perform classification on a higher-dimensional dataset.

Pernek et al. \cite{dtw} propose an algorithm to count the number of repetitions of an exercise using DTW, a dynamic programming technique which allows for comparison between two non-temporally aligned signals by calculating a mapping which minimally warps and shifts one signal onto another. To differentiate between exercise and non-exercise, Pernek et al. utilize a thresholding algorithm which triggers when the device's accelerometer signal peaks approach the magnitude of the peaks in their prerecorded dataset. Their method performs very well with regard to repetition count, although their solution is not entirely autonomous during operation, requiring input from the user at the beginning of each exercise.

Seeger et al. \cite{seeger} describe a system which utilizes a network of embedded wearable sensors across the body to compute high-dimensional features for exercise classification. Equipping a user with an accelerometer above the right knee, a heart rate sensor, an accelerometer attached to a weight lifting glove, and a chest strap, this system is able to highly accurately detect and count exercise. However, this system is suboptimal due to the infrastructure required. A user attending an incredibly upscale gym may have access to these sensors, but the average user would not. Wearing so many sensors would also obstruct the user during lifting, which could cause both damage to the sensors and discomfort to the user. 

Muehlbauer et al. \cite{muehlbauer} follows a similar pattern to Morris et al. \cite{recofit} and our own solution by dividing the task into three phases, segmentation, recognition, and counting. Autocorrelation analysis is used during segmentation to determine of a user is performing an exercise. After determining that an exercise is being performed, a number of features are calculated such as mean and standard deviation. These are passed into a KNN classifier which comprises the recognition phase. Lastly, counting is perfored using simple peak counting. Muehlbauer et al. performs well, with 85\% segmentation accuracy and 94\% recognition accuracy, although their segmentation thresholds are based off heuristics, and they do not address online performance. 

To summarize, the discussed related systems often fall short in one key category. For our solution, we aim to track exercises using no additional hardware beyond the smartwatch, without user input during the exercise session, and in an online fashion, all while maintaining high accuracy rates. 