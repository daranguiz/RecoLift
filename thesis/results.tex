\chapter{Results}


\section{Methods}
Our accuracy can be characterized by three factors:

\begin{enumerate}
    \item \textbf{Segmentation}: Does the system properly detect \textit{when} a lift is being performed? Additionally, does the system ignore all idle motion?
    \item \textbf{Recognition}: Does the system properly determine \textit{which} lift is being performed?
    \item \textbf{Counting}: Does the system count the number of repetitions correctly?
\end{enumerate}

To test our system, we trained our classifiers on five lifts: low bar squat, overhead press, bench press, Pendlay row, and barbell curl. Testing comprised of performing 5 sets of each lift at 8 repetitions each, totaling 40 repetitions for each lift. An empty barbell weighing 45 lbs was used. 

\begin{table}
    \centering
    \begin{tabular}{l r r r}
        \toprule
        Lift           & Exact   & Within 1 & Within 2 \\
        \midrule[\heavyrulewidth]
        Low Bar Squat  & 80 \%    & 80 \%     & 100 \% \\
        Overhead Press & 40 \%    & 100 \%    & 100 \% \\
        Bench Press    & 40 \%    & 40 \%     & 80 \%  \\
        Pendlay Row    & 40 \%    & 80 \%     & 100 \% \\
        Curl           & 20 \%    & 100 \%    & 100 \% \\
        \midrule
        Overall        & 44 \%   & 80 \%      & 96 \%  \\
        \bottomrule
    \end{tabular}
    \caption{Difference in reported repetition count versus actual repetition count}
\end{table}

\section{Final Results}
We achieved 100\% accuracy both on segmentation and recognition. Every set performed was noted and classified correctly, and no misclassification of idle movement occurred. The watch was given to a different user at the gym for a single set, however, and that set was misclassified. Perfect consistency with one user and variable accuracy with another user implies a lack of training data; in the future, our dataset will be trained on more than a single user.

Recognition achieved permissible levels of accuracy, with 80\% of sets being reported within one repetition of their true count. 

Considering this system was physically implemented, we can make inferences regarding battery life. The Moto 360 smartwatch used in this system lost approximately 25\% of its battery life during our one hour testing session, and the Galaxy S4 smartphone used in this system lost approximately 15\% of its battery life. Further user study will determine whether this is permissible or not; conserving battery life is an important factor for many in the mobile space, and a two hour gym session draining 50\% of the watch's battery may not be acceptable. Future iterations of Android Wear devices should allay battery life concerns, however.